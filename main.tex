\documentclass[12pt]{article}

% Geometry and fonts
\usepackage[left=1in,top=1in,right=1in,bottom=1in]{geometry}
\newcommand*{\authorfont}{\fontfamily{mathpazo}\selectfont}
\usepackage[]{mathpazo}
\usepackage[T1]{fontenc}
\usepackage[utf8]{inputenc}

% Languages
\usepackage[spanish,english]{babel}

% Math, tables, figures
\usepackage{graphicx}
\usepackage{amssymb}
\usepackage{amsbsy}
\usepackage{amsmath}
\usepackage{bm}
\usepackage{amsthm}
\usepackage[table]{xcolor}
\usepackage{dcolumn}
\usepackage{longtable}
\usepackage{multirow}
\usepackage{booktabs}
\usepackage{tabularx}
\usepackage{siunitx}
\usepackage{listings}
\newcolumntype{d}{S[input-symbols = ()]}
\usepackage{threeparttable}
\usepackage{rccol}
\usepackage[hang]{caption}
\usepackage[figuresright]{rotating}
\usepackage{lscape}
\usepackage[tableposition=bottom]{caption}

% Hyperref — keep BEFORE biblatex
\usepackage[colorlinks,linkcolor=purple,
citecolor=teal,urlcolor=purple,
bookmarks=false,hypertexnames=true]{hyperref}

\usepackage{xr-hyper}
\externaldocument{appendix}
\usepackage{xurl}
\usepackage{tikz}
\usetikzlibrary{positioning}
\usetikzlibrary{arrows.meta}
\usetikzlibrary{arrows,intersections}
\usepackage{csquotes}

% Replace natbib with biblatex-chicago (natbib removed)
\RequirePackage[authordate-trad,noibid,backend=biber,natbib]{biblatex-chicago}

% Your .bib files
\addbibresource{References.bib}

% Clean fields to keep bibliography short
\DeclareSourcemap{
  \maps[datatype=bibtex]{
    \map{
      \step[fieldset=issn, null]
      \step[fieldset=isbn, null]
      \step[fieldset=language, null]
      \step[fieldset=urldate, null]
      \step[fieldset=month, null]
      \step[fieldset=note, null]
      \step[fieldset=doi, null]
      \step[fieldset=url, null]
    }
  }
}

% Clean "note" field in final bib
\AtEveryBibitem{%
  \clearfield{note}%
}

% Only print years for \citeyear
\DeclareCiteCommand{\citeyear}
    {\usebibmacro{prenote}}
    {\bibhyperref{\printfield{year}}\bibhyperref{\printfield{extrayear}}}
    {\multicitedelim}
    {\usebibmacro{postnote}}

% Remove month/season etc.
\renewbibmacro*{clear+datefield}[1]{%
  \clearfield{#1year}%
  \clearfield{#1month}%
  \clearfield{#1season}%
  \clearfield{#1endyear}%
  \clearfield{#1endmonth}%
  \clearfield{#1endseason}%
}

\usepackage{setspace}
\usepackage{csquotes}


\begin{document}

\title{Class voting versus economic voting: Explaining electoral behavior before and after the `Left turn' in Latin America}
\author{
Rosario Queirolo\thanks{Universidad Católica del Uruguay. }
\and
Santiago López-Cariboni\thanks{Universidad de la República }
}
\date{\today}

\maketitle
\thispagestyle{empty}
\vspace{-1em} % tighten spacing after title

\begin{abstract}
\noindent
It is common wisdom that class-based cleavages are electorally weak in Latin America, and other explanations such as economic voting receive stronger empirical support. Between 1970s and 1990s many left wing opposition parties succeeded in introducing a new ideological dimension in electoral competition based on the support of privileged and educated voters. After the `left turn' some of these parties have implemented a number of innovative pro-poor policies and cultivated the vote of the disadvantaged. There is still little knowledge on whether Latin American ``pink tide'' contributed to the transformation of electoral politics toward a more class-based competition. This paper explores the possibility that the relative weights of economic voting and class-based voting have changed after two decades of leftist governments in Latin America.  Empirically, we estimate a number of voting behavior models to assess the evolution of the effects of macroeconomic perceptions and social class over the course of eight elections between 1989 and 2024 in Uruguay.
\end{abstract}

\clearpage
\setcounter{footnote}{0}
\doublespacing



\section*{Introduction} % (fold)
\label{sec:introduction}



Class voting has been never considered a relevant explanation of Latin Americans political behavior. Among the usual suspects to predict how citizens cast their vote in the region, economic evaluations and ideological self-placement take the lead. For instance, the shift to the Left during the last two decades has been explained mainly as an outcome mandate, where popular economic discontent with the previous governments played a crucial role (Arnold and Samuels 2011; Murillo, Oliveros, and Vaishnav 2011; Queirolo 2013). But the temporal persistence of `left turn' during several elections is in many aspects problematic for the argument that class-based parties are rare in Latin America since electoral platforms are non-binding contracts and political parties are unable to credible commit with policies. This is a common place in the literature. Yet, simple recognition of the steady growth of left-wing programmatic parties hat occurred during the first decade of the 2000, promoting redistribution for the poor together with the absence of class voting behavior reveals that an important theoretical and empirical puzzle remains unsolved. Existing research on linkages between parties and citizens in the Latin America has devoted much of its attention to aspects of living standards and economic voting, ideology, clientelism, and political institutions. Quite surprisingly though, research on class-voting is a largely neglected topic in a continent that experienced a sea-change toward governments with redistributive platforms. Hence, a fundamental problem is understanding the impact of left-of-center governments on the amount of class voting and its balance against alternative forms of vote choice also based on material considerations, namely, economic voting.

Economic voting and class voting are two different mechanisms of individual political behavior. The former is a valence issue, where individuals use economic information to judge how competent is the government to deal with the economy. The later refers to whether income groups have homogeneous electoral preferences for political parties that represent their distributive interests. It is possible that after a decade of experiencing the implementation of pro-poor policies by leftist governments, Latin Americans started following class alignments more consistently than they did before. 

First, this paper opens the possibility that class voting may have counterbalanced the amount of economic voting during the left turn. We propose the very simple argument that when redistributive parties commit with their electoral platforms---i.e., left-wing parties that deliberatively favor the poor once in office---class voting is more likely to happen. While we do not necessarily contradict much of the previous research on electoral behavior and political representation in Latin America, we make the important addition that commitment with redistributive electoral platforms significantly aligns individuals' position in the income distribution and party vote choice.\footnote{This by no means explains successful new-party entry and emergence of left-wing challenging parties in Latin America. But see, for instance, \cite{Lopez-Cariboni2005}: successful new party-entry depends on the credible commitment of left-wing parties to stand for their electoral platforms, which in this case, is expressed by not entering into coalition government with parties of the status quo.}

Second, given that incumbent parties not only deliver policy but also produce information, we also analyze the consequence of left-wing governments for economic voting itself across income groups. Groups may change their use of macroeconomic data to reward-punish the incumbent if they receive new different signals regarding the distributive fairness of the observable aggregate growth. For instance, an important result of democratic politics in the U.S. is that voters tend to reward governments for the income growth of the rich besides the average income growth (Bartels 2008). This phenomenon of ``class-biased economic voting'' results from the fact that electoral politics and governments manipulate economic information inducing the idea that the growth of the rich is a good signal for all income groups. The extent to which middle-class and low-class voters reward the incumbent for the unequal distribution of economic growth reflects the amount of class-bias in economic voting. We argue that left parties that commit to redistribution are likely to attenuate the class-bias of economic voting that favors the growth of top income earners.


Using income growth and survey data from 8 elections in Uruguay between 1989 and 2024 we find that, a) the poor consistently vote for the left since 2009, this is, after the first left-wing government, even after controlling for economic voting, ideology and party identification; b) but that effect diminished in 2019 and dissipated in 2024;  c) economic voting has marginally decreased over time; c) there was a considerable amount of class-biased of economic voting among middle- and low-class individuals in favor of the rich until 1999; d) but after 2004 the bias in favor the top earners became negative among middle-class individuals and insignificant among low-class individuals; e) low-class voters did not reward the left government for the income growth of the bottom 10\% above the average income growth. Taken together these results give support to our argument that class-voting may have emerged after the left turn and is distinguishable from economic voting. Moreover, economic voting has changed from the class-biased logic described by Bartels (2008) for the U.S case, to a new logic suggesting a lower electoral advantage due to the growth of top income earners. 

However, class-voting started to diminish since 2019, and in 2024 was no longer present among Uruguayans. It is not new that explanations to voting behavior depend on the election and are not always the same (ADD REF), and class voting is not the exception. During one decade, from 2009 to 2019, class voting was important among voters in Uruguay. In this period, the Frente Amplio was in charge of the government, and the country experienced economic prosperity partly as a result of the commodity boom. When prosperity winds ends, class voting vanished. 


% Bartels' argument has not been tested in Latin America, and the region has become a nice setting to probe it due to the prevalence of left oriented governments.

% Arguably, disentangling economic and class voting may be theoretically and empirically difficult. Whose economic fortune is the one that voters take into account? We show that another consequence of left-wing governments in Latin is a change in the class-bias of economic vote. 
 

The paper is organized as follows. The next, and second, section describes how the economic voting and class voting work and the evidence that support them in Latin America. The third section states the arguments to be tested. The fourth section describes the research design, method and data. Results are presented in section fifth and concluding remarks at the end. 



\section*{Economic voting and class voting } % (fold)
\label{sec:economic_voting_and_class_voting_}
Two of the most powerful explanations to voting behavior are the economic voting theory and the class cleavage theory. Economic voting theory states that people vote taking into account the performance of the economy. Voters that have positive evaluations of the economy will reward the government by voting for the incumbent party, while those who have negative evaluations will punish the incumbent government and vote for the opposition (Fiorina, 1981; Lewis Beck 1982, 1986, 1988). Economic assessments can take at least four different forms by combining time (MacKuen, Erikson, and Stimson, 1992) and personal/national impact (Kinder and Kiewiet, 1981). First, evaluating how good or bad the economic situation of the country has been during the past (Retrospective Sociotropic); second, taking into account their expectations of how the country's economic situation is going to be in the future (Prospective Sociotropic); third, thinking on how good or bad their family's economic situation has been in the recent past (Retrospective Pocketbook); and finally, considering their expectations for their family's economic future (Prospective Pocketbook).

In each of these forms, the economic voting theory has markedly proved its predictive power in developed countries \citep{(Lewis-Beck 1988; Lewis-Beck and Stegmaier 2013; Nadeau and Lewis-Beck 2001; Nadeau et al 2013; Vavreck 2009). In particular, evidence is quite supportive on showing that Europeans and Americans voters are more responsive to changes in general economic conditions than to their personal economic circumstances (Kinder, 1998). 

Despite some summaries of economic voting theory have introduced doubts on the power that this political behavior has on generating democratic accountability (Anderson 2007) or the influence that contextual variables have on the relevance of economic voting (Duch and Stevenson 2008), economic assessments are crucial voting predictors. 

Latin Americans are not different from their counterparts in the most established democracies. Studies indicate that they favor the incumbent when the economy goes well, and punish it when economic downturns prevail \citep{Echegaray, gelineau2013electoral,murillo2017economic,Remmer 1993; Roberts y Wibbels, 1999, singer2015electoral, valdini2018economic}. After four decades of continuous democracy in almost all the region, voters behave as rational actors by making their political leaders accountable for the economic situation. The idea of Latin Americans as highly volatile and unpredictable voters, easily prone to clientelism (Remmer, 1993; Ames, 2001), is fading.

On the other side, class cleavage theory explains voting behavior by looking at the socioeconomic position that people have in society.  Lipset and Rokkan (1967) stated this theory arguing that party systems in Western Europe were so stable because political parties based their historical roots on class, religion, and nationality differences. Following this reasoning, citizens would vote depending on the position occupied in society. Regarding social class, workers would tend to vote for left oriented political parties while bourgeoisie people vote for right of center parties. 

Contrary to the economic voting theory, social class cleavage theory is not popular among voting behavior researchers in Latin America. Several scholars have tested the argument but found no evidence confirming the existence of class voting \citep{(Etchegaray, 2005; Klesner, nadeau2017latin, Queirolo, 2013,    Roberts and Wibbels, 1999;  singer2015electoral}. In cases where there is some proof; it is mainly related to one particular country in an identified election, not a general voting cleavage. For example, Chile before Pinochet's dictatorship but not after the transition to democracy \citep{Roberts and Wibbels, bargsted2016social, mainwaring2004class}, Venezuela under the government of Hugo Chávez  \citep{lupu2010votes, Roberts, 2003a} and a couple of elections in Argentina (Cantón and Jorrat, 2002).  

Recent research on economic voting theory states hypotheses integrating economic voting with changes in income distribution. In a way, this new offspring that combines economic voting with class, or as Bartels (2008) named it the ``class biased economic voting'' (CBEV), is focused on answering the question of whose economy voters have in mind when evaluating the incumbents.  Mutz and Mondak (1997) notice that voters would reward the incumbent party when they perceive that class groups have experienced similar, rather than dissimilar, changes in economic performance. Bartels found a different phenomenon, a class biased in economic voting, one that favors upper class trajectories: ``Americans voters, regardless of their own place in the income distribution, seem to be quite sensitive to the economic fortunes of high-income families but much less sensitive to income growth among middle-class and poor families'' (2008: 111). Hopkins (2012) contradicts Bartels argument and shows that income's grow among the poor can be ever more influential on voters behavior than income grow among the rich.  However, all these arguments were only tested in the United States. 

Most recently, Hicks, Jacobs, and Matthews (2014) proved Bartels argument in 15 advanced industrialized democracies and found that electorates, in general, are more responsive to top-income growth than to the mean-income growth. Their analysis on elections in Sweden,  Canada,  and  the  United  States discover that also in Sweden and Canada, countries that differ from the US in their cultural and institutional conditions, low  and  middle-income  voters are more sensitive to the fortunes  of  those at  the  very  top  of  the  income  scale.
The existence of a class-biased in economic voting towards the wealthiest ones is a problematic finding because it could reinforce income inequalities. First, a class-biased in economic evaluations strengthens the electoral fortunes of political parties that favor upper class sector. Second, it can also increase the incentives of any political party, regardless of their prior electorate, to privilege the rich because the payoffs are greater than benefiting the poor, or those that belong to the middle sectors (Hicks, Jacobs and Matthews, 2014). Finally, this class-bias clearly undermines the idea that economic voting was a powerful mechanism to make politicians, and democracy more broadly, accountable.
The class-biased economic voting argument has not been tested in Latin America, but the region has become an interesting setting to assess it due to the prevalence of left wing governments since the beginning of the twenty first century. Starting in 1998 with Hugo Chávez reaching the presidency in Venezuela, the ``pink tide'' expanded in the region: Chile in 1999, Brazil in 2002, Argentina in 2003, Uruguay in 2004, Bolivia in 2005, Nicaragua, Peru and Ecuador in 2006, Guatemala in 2007, Paraguay in 2008 with a very shortly experience, El Salvador in 2009, and Costa Rica in 2014. Behind this shift is an outcome mandate, a claim of Latin Americans for better living standards, and underneath the outcome mandate is economic voting.

In one way or the other, these Latin American countries have experienced governments with a leftist orientation supposed to favor the interest of the working class or the poor. In most of them, incumbents' parties have been reelected. Is this an indicator that Latin Americans are behaving differently from their counterparts in most developed countries and recompensing political parties that make the poorest to progress? This would contradict Hicks, Jacobs and Matthews (2014) argument that country different conditions do not matter. 

% section economic_voting_and_class_voting_ (end)


\section*{Argument} % (fold)
\label{sec:argument}

% section argument (end)
The idea that a class cleavage could have emerged in Latin America after the ``pink tide'' is based on the argument that redistribution to the poor increases the amount of class voting among those who benefit from redistribution. In other words, credible policy commitment with redistribution from the part of the incumbent provides a number of resources and information that may contribute to the emergence of class voting among low class individuals.  Class identity and economic performance may be competitive mechanisms of vote choice, so we explore the possibility that redistribution in favor of the poor decreases the amount of economic voting hence the relative weight of class voting. Following this argument, our hypothesis is that, regardless of the relevance of economic voting, class identity should become a significant voting mechanism within Latin Americans after a government compromised with redistribution. In other words, we should find an emerging class cleavage after this decade of left wing governments in the region.   

Our second  argument considers the idea that economic voting and class voting interact to explain the vote for the incumbent. Income groups may be sensible to the distributive dynamics of economic growth due to the information that left-wing governments disseminate. This is, economic voting itself may depend on the degree to which top income earners are able to appropriate the benefits of economic growth. This opens the question of whether economic voting is biased in favor of unequal economic growth (Bartels, 2008). Our main intuition is that the middle class voters are those who align their responsiveness to economic conditions with their preferences for redistribution. If pivotal actors selected a redistributive government, which in turn disseminates information about fair economic growth, then responsiveness to economic growth is likely to reflect inequality-aversion. If, however, median actors have selected a government not committed with redistribution, it is less likely that they punish relatively larger gains among the very rich. This is, class-biased economic voting should be driven by whether the redistributive preferences of the median voter are met by the incumbent party. 

Following this argument, we hypothesize that leftist incumbents should be penalized if top income earners growth increase faster than the average, while right wing incumbents do not. On the contrary, leftist governments should be rewarded when low class voters growth rate is higher than the average. 


\section*{Data and Methods}
\label{sec:data_and_methods}

We test our arguments using evidence from Uruguay, a theoretically valuable case for assessing the emergence of class-based voting. Prior to the victory of the center-left \emph{Frente Amplio} in 2004, research consistently concluded that class cleavages were weak or absent, whereas economic evaluations strongly shaped electoral behavior (Luna 2002; Queirolo 2013). Uruguay’s traditional parties, the \emph{Partido Colorado} and \emph{Partido Nacional}, historically operated as catch-all organizations (Gillespie 1986; González 1991), with their support structured around urban--rural, demographic, and partisan identities rather than socioeconomic position. The \emph{Frente Amplio}, founded in 1971, initially drew support from urban, educated, and middle-class voters but progressively expanded into regions and social segments previously dominated by the traditional parties (Moreira 2005).

The consolidation of left-wing governance after 2004 raises a central question for theories of democratic representation in Latin America: did the implementation of redistributive policies under the \emph{Frente Amplio} generate an observable class cleavage in voting behavior? In particular, did lower-income voters begin to support the left independently of economic conditions or partisan attachments? While prior work offers suggestive evidence (Moreira 2010; Bogliaccini and López-Cariboni 2015), a systematic assessment across multiple electoral cycles remains lacking. To address this gap, we assemble survey data from multiple election years: 1989 (Equipos Mori), 1994--2004 (CIFRA), 2009 and 2014 (AmericasBarometer), and 2024 from CIFRA’s publicly released electoral opinion pool.\footnote{The 2024 survey was collected and published by CIFRA as part of its national pre-electoral polling series.} Pooling these sources produces a harmonized repeated cross-section spanning 1989--2024. Key measures were recoded to ensure comparability across surveys.

Our measure of economic voting relies on a standard sociotropic question asking respondents to evaluate the country’s current economic situation; answers of ``good'' or ``very good’’ are coded as favorable evaluations. Because income variables differ across surveys, we rely on three operationalizations of class: (1) continuous household income per capita; (2) income quintiles; and (3) terciles approximating lower-, middle-, and upper-class positions.\footnote{Income serves as our primary class indicator both for consistency with the class-biased economic voting literature (Bartels 2008) and because comparable occupational measures are unavailable across survey years.} Two additional predictors capture central mechanisms of voting behavior. First, ideological self-placement distinguishes voters who prioritize redistribution from those who favor market-oriented policies. Second, identification with the incumbent party accounts for the strong partisan anchoring common in presidential systems. To assess how the relevance of these mechanisms evolves over time, we interact each predictor with election-year indicators, effectively estimating separate slopes for each election.

A central feature of our empirical strategy is that we fit \emph{two parallel logistic regression models}, which share the same predictors and interaction structure but differ in their dependent variables. In the first model, the outcome is \emph{incumbent vote}, the conventional measure in studies of economic voting. This specification makes it straightforward to assess how sociotropic evaluations translate into electoral rewards and punishments. In the second model, the outcome is \emph{left vote}, which allows us to trace the emergence of class alignments with the \emph{Frente Amplio} independently of whether the left was the incumbent in a given election. Although the outcomes differ, the models are analytically equivalent: each recovers the conditional association between income, economic perceptions, ideology, and partisan identification and support for the party of interest in each election year. Presenting both outcomes simply highlights different theoretical dimensions---economic accountability in the incumbent-vote model and class realignment in the left-vote model.

Formally, for the incumbent-vote specification, we estimate:
\begin{align*}
\Pr(\text{Incumbent Vote}_i = 1)
&= \text{logit}^{-1} \Bigg[
      \sum_{k=1}^{4} \sum_{t} \beta_{kt}\,
        (Q_{ki} \times \text{Year}_{it})
\\ &\qquad
    + \sum_{t} \alpha_{t}\,
        (E_i \times \text{Year}_{it})
\\ &\qquad
    + \sum_{t} \gamma_{t}\,
        (I_i \times \text{Year}_{it})
\\ &\qquad
    + \sum_{t} \delta_{t}\,
        (P_i \times \text{Year}_{it})
\Bigg],
\end{align*}

where $Q_{ki}$ is an indicator for income quintile $k$, $E_i$ denotes sociotropic economic evaluations, $I_i$ represents ideological self-placement, $P_i$ identifies partisanship toward the incumbent party, and $\text{Year}_{it}$ is a dummy for election year $t$. The left-vote model shares this exact structure with the dependent variable replaced by $\Pr(\text{Left Vote}_i = 1)$.

This flexible interaction structure allows the effects of economic evaluations, ideology, partisanship, and class to vary freely across elections, avoiding assumptions of temporal homogeneity. Importantly, the coefficients $\beta_{kt}$ identify the \emph{partial association between income group membership and support for the party of interest in each election year}, net of economic perceptions, ideological orientation, and partisan identification. Statistically significant quintile coefficients therefore constitute evidence of class voting that cannot be attributed to the dominant explanatory frameworks in the literature.


To clarify this strategy, Figure~\ref{fig:OA-dag} in the Supplementary Materials (Section~\ref{sec:dag}) presents a directed acyclic graph (DAG) summarizing the model structure. The DAG highlights the direct pathways from income, economic evaluations, ideology, and partisanship to vote choice, the endogenous associations among these predictors, and the role of election year as an exogenous modifier of all effects. This visualization mirrors the interaction structure of our empirical models and provides a conceptual roadmap for interpreting the estimated coefficients.



\section*{Results}
\label{sub:results}
 We begin by evaluating the persistence of economic voting over the 1989--2024 period. Appendix~\ref{sec:appendix-econ} and Figure~\ref{fig:appendix-econ} show that positive sociotropic evaluations remain a strong and consistently significant predictor of vote choice. Although the magnitude of economic voting fluctuates---peaking in 2004 in the aftermath of the 2002 crisis and declining somewhat in subsequent elections—it does not disappear. Full coefficient tables for both the incumbent-vote and left-vote models appear in Appendix~\ref{sec:appendix-tables}. Overall, Uruguay exhibits durable economic accountability throughout the democratic period. We now turn to our main interest: the evolution of class-based voting. Figure~\ref{fig:logit_quintile_effects_left} presents the estimated effects of income quintiles on support for the Left, relative to the richest quintile (Q5). These results come from the fully specified interaction model introduced in the research design section, and the corresponding tables are reported in the Supplementary Materials. To facilitate interpretation, we display the results as a \emph{threeparttable} that highlights confidence intervals and election-by-election patterns.


\begin{figure}[htbp]
\centering
\begin{threeparttable}
\includegraphics[width=0.9\textwidth]{Figures/figure_logit_quintiles_left.pdf}
\begin{tablenotes}
\footnotesize
\item \textit{Note:} Coefficients are from election-specific logistic regressions of Left vote on income quintiles, sociotropic economic evaluations, ideology, and incumbent partisanship. Confidence intervals are based on standard errors clustered by survey year. Shaded intervals in the figure correspond to 95\% confidence bands. Years in which a quintile coefficient excludes zero indicate statistically significant differences from the richest quintile.
\end{tablenotes}
\end{threeparttable}
\caption{Effects of income quintiles (Q1--Q4, baseline Q5) on the probability of voting for the Left across Uruguayan elections, 1989--2024. Estimates are logit coefficients with 95\% confidence intervals. Positive (negative) coefficients indicate greater (lower) support for the Left compared to the richest quintile.}\label{fig:logit_quintile_effects_left}
\end{figure}


The pattern in Figure~\ref{fig:logit_quintile_effects_left} documents a clear cycle in Uruguay’s class cleavage: its \emph{emergence}, \emph{consolidation}, and \emph{attenuation}. From 1989 to 2004, class voting is effectively absent. Point estimates for Q1--Q4 are small and statistically indistinguishable from zero, indicating that income did not meaningfully structure support for the Left before its arrival in power.

A sharp shift occurs after the Left's historic victory in 2004. In the 2009 and 2014 elections—two contests in which the \emph{Frente Amplio} governed as incumbent—coefficients for the poorest quintiles (Q1 and Q2) turn strongly positive and statistically significant. These groups became substantially more likely than high-income voters to support the Left. Middle-income voters (Q3) also exhibit positive shifts, though with less precision. This marks the consolidation of a redistributive alignment: lower-income voters disproportionately supported the Left during its period of governance.

The pattern weakens but does not reverse after the right regained power in 2019. Estimates for 2019 and 2024 fall toward zero and often lose statistical significance—yet remain consistently above the pre-2004 baseline. These elections suggest that the class cleavage did not disappear with partisan turnover: poorer voters remained more favorable to the Left even when it was no longer the incumbent, but the magnitude of the effect was smaller than during periods of left-wing governance.

To assess the substantive magnitude of these shifts, Appendix~\ref{sec:appendix-class} reports the associated changes in predicted probabilities (Figure~\ref{fig:probability_change_income_quintiles_left}). These results confirm the pattern observed in the coefficients. In 2009 and 2014, moving from Q5 to Q1 increases the probability of supporting the Left by sizable margins, whereas in 1989, 1994, and 1999 the same contrast yields effects close to zero. The 2019 and 2024 elections show intermediate values: class voting persists, but its strength declines compared to its peak under left incumbency.

Finally, Appendix~\ref{sec:appendix-class} also presents complementary models in which the outcome is support for the incumbent, regardless of ideology. These results corroborate the left-vote pattern. When the Left governs (2009, 2014), poor voters (Q1 and Q2) exhibit higher relative support for the incumbent; when the right governs (2019, 2024), these same groups withdraw support. This asymmetric pattern shows that class voting is driven not by incumbency per se, but by the ideological orientation of the incumbent party.

Taken together, our results reveal a distinct temporal structure in Uruguay’s class cleavage. Economic voting remains a durable feature of electoral politics, but class voting emerges only after the Left's entry into government, peaks during its consolidation in power, and declines once partisan control shifts—while still remaining above historical levels. The Uruguayan case thus illustrates how distributive policy, partisan alternation, and voter alignment can dynamically reshape class cleavages over time.


\section*{Concluding Remarks} % (fold)
\label{sec:concluding_remarks}
Evidence for the Uruguayan case shows that in Latin America, after the shift towards the Left, there is an emerging class cleavage.  Class voting exists among poor voters after the first government of the leftist Frente Amplio, and coexists with the well- established economic voting. 

The existence of economic voting is generally considered good news because makes politicians and political parties accountable (Lewis-Beck and Stegmaier 2008). Incumbents' parties are reelected when they had provided economic well being to their constituencies. On the other side, class voting refers to representation. Citizens make their voting decisions by favoring the political party or candidate that better represent their interests as a class. The Uruguayan evidence presented in this paper indicates that  access of left wing parties to the governments of Latin America have started to develop a class cleavage, one that was absent before the ``left turn''. This is also clear from comparing  the class biased of economic voting before and after the arrival of the left to the national government. 

However, two caveats to these conclusions are necessary. First, the final confirmation of the emergence of class voting requires a change in the ideology of the government. In other words, only if low class voters keep voting leftist parties regardless that a right wing party is the incumbent and country's economy goes well, we can be sure that an enduring class cleavage was developed. Second, we should test this argument to the whole region. 







\pagebreak
\printbibliography

\newpage
\appendix


\end{document}